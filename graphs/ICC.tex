% !TEX program = xelatex


\documentclass{../Jorrearticle}

\usepackage{tabularray}
\usepackage{tikz}
\input{tikzset.tex}

\usepackage[active,tightpage]{preview}
\newenvironment{previewoutput}{}{}
\PreviewEnvironment{previewoutput}
\setlength{\PreviewBorder}{0pt} 


\begin{document}






\tikzset{y=4mm}


\newcommand\yaxis{%
	\begin{tikzpicture}[baseline=0,trim right=0,remember picture]
	\draw[black!50] (0,0) -- (0,13);
	\foreach \y/\l in {0/0,5/5,10/10}{
		\draw[black!50] (0,\y) -- +(-2mm,0) node(S)[yaxislabel]{\l\strut};
		}
	\node[axistitle,black!50,anchor=south west,xshift=-10mm,align=left] at (0,13) 
		{Zuurstofgehalte (mg/L)};			
	\end{tikzpicture}%
	}
		
	
\newcommand\yaxisb{%
	\begin{tikzpicture}[baseline=0]
	\node[plotlabel,black!30,anchor=west,align=left,xshift=2pt] at (0,9.016652) {gemiddelde = 9.02};	
	\node[plotlabel,red!40!white,anchor=west,align=left,xshift=2pt] at (0,10.5) {grenswaarde = 10,5};
	\end{tikzpicture}%
	}	
	
	
\newcommand\xaxis{%
	\begin{tikzpicture}[x=\linewidth/12,trim left=0,trim right=\linewidth]
	\draw[black!50] (0,0) -- (12,0);
	\foreach \x/\l in {0/j,1/f,2/m,3/a,4/m,5/j,6/j,7/a,8/s,9/o,10/n,11/d}{
		\node at (\x.5,0) [xaxislabel,align=center]{\l};
		}
	\end{tikzpicture}%
	}

\newcommand\body{%
	\begin{tikzpicture}[x=\linewidth/12]
	\useasboundingbox (.5,0) rectangle (12.5,13);
%	\foreach \y/\l in {0/0,1/10\,000,2/20\,000,3/30\,000,4/40\,000,5/50\,000}{
%		\draw[mistgray] (0,\y) -- (11,\y);
%		}
	\draw[black!30] (.5,9.016652) -- +(12,0);
	\draw[red!40!white] (.5,10.5) -- +(12,0);
	\foreach \x/\y in {1/7.2458,1/7.4306,2/8.6288,2/8.0061,3/9.0389,3/9.2460,4/10.8394,4/10.0922,5/10.2332,5/11.1516,6/11.0843,6/11.2657,7/11.1105,7/10.0287,8/9.6652,8/9.8641,9/8.7950,9/8.9231,10/7.6614,10/7.3010,11/7.7109,11/7.2852,12/7.1870,12/6.6051}{
		\fill (\x,\y) circle(2pt);
		}
	\draw (1,7.3382) -- (2,8.3174) -- (3,9.1424) -- (4,10.4658) -- (5,10.6924) -- (6,11.1750) -- (7,10.5696) -- (8,9.7646) -- (9,8.8590) -- (10,7.4812) -- (11,7.4981) -- (12,6.8960);
	\end{tikzpicture}%
	}


\begin{figure}[h]
\begin{previewoutput}
%\graphtitle{Testresultaten met relatief lage variatie tussen labotesten en dus hoge ICC.}
\begin{tblr}{
	width=\linewidth,
	colspec={rX[c]r},
	column{1}={leftsep=0pt},
	column{Z}={rightsep=0pt,leftsep=0pt},
	rows={rowsep=0pt},
%	columns={colsep=0pt},
%	stretch=0,
	}
\yaxis & \body & \yaxisb \\ 
& \xaxis &  \\ 
\end{tblr}
\end{previewoutput}
\end{figure}












\renewcommand\body{%
	\begin{tikzpicture}[x=\linewidth/12]
		\useasboundingbox (.5,0) rectangle (12.5,13);
		%	\foreach \y/\l in {0/0,1/10\,000,2/20\,000,3/30\,000,4/40\,000,5/50\,000}{
			%		\draw[mistgray] (0,\y) -- (11,\y);
			%		}
		\draw[black!30] (.5,9.016652) -- +(12,0);
		\draw[red!40!white] (.5,10.5) -- +(12,0);
		\foreach \x/\y in {1/8.1634,1/6.3838,2/9.4493,2/7.4728,3/9.8028,3/8.4612,4/10.7281,4/8.6904,5/9.5373,5/11.7475,6/12.5272,6/11.9835,7/11.9810,7/8.2369,8/10.8738,8/7.5407,9/9.8104,9/9.5154,10/8.8871,10/6.1224,11/6.6643,11/8.1963,12/7.6120,12/6.0121}{
			\fill (\x,\y) circle(2pt);
			}
		\draw (1,7.2736) -- (2,8.4611) -- (3,9.1320) -- (4,9.7093) -- (5,10.6424) -- (6,12.2554) -- (7,10.1089) -- (8,9.2073) -- (9,9.6629) -- (10,7.5047) -- (11,7.4303) -- (12,6.8120);
	\end{tikzpicture}%
}


\begin{figure}[h]
	\begin{previewoutput}
%		\graphtitle{Testresultaten met relatief hoge variatie tussen labotesten en dus lage ICC.}
		\begin{tblr}{
				width=\linewidth,
				colspec={rX[c]r},
				column{1}={leftsep=0pt},
				column{Z}={rightsep=0pt,leftsep=0pt},
				rows={rowsep=0pt},
				%	columns={colsep=0pt},
				%	stretch=0,
			}
			\yaxis & \body & \yaxisb \\ 
			& \xaxis &  \\ 
		\end{tblr}
	\end{previewoutput}
\end{figure}






\end{document}

		

