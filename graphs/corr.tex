% !TEX program = xelatex


\documentclass{../Jorrearticle}

\usepackage{tabularray}
\usepackage{tikz}
\input{tikzset.tex}

\usepackage[active,tightpage]{preview}
\newenvironment{previewoutput}{}{}
\PreviewEnvironment{previewoutput}
\setlength{\PreviewBorder}{0pt} 


\begin{document}






\tikzset{y=50mm}


\newcommand\yaxis{%
	\begin{tikzpicture}[baseline={(0,7.8)},trim right=0,remember picture]
	\draw[black!50] (0,7.8) -- (0,8.8);
	\foreach \y/\l in {8.0/8.0,8.2/8.2,8.4/8.4,8.6/8.6}{
		\draw[black!50] (0,\y) -- +(-2mm,0) node(S)[yaxislabel]{\l\strut};
		}
	\node[axistitle,black!50,anchor=south west,xshift=-10mm,align=left] at (0,8.8) 
		{pH-waarde};			
	\end{tikzpicture}%
	}
		
	
\newcommand\yaxisb{%
%	\begin{tikzpicture}[baseline=0]
%	\node[plotlabel,black!50,anchor=west,align=left,xshift=2pt] at (0,9.016652) {gemiddelde = 9.02};	
%	\node[plotlabel,red!80!black,anchor=west,align=left,xshift=2pt] at (0,10.5) {grenswaarde = 10,5};
%	\end{tikzpicture}%
	}	
	
	
\newcommand\xaxis{%
	\begin{tikzpicture}[x=\linewidth/50,trim left=0,trim right=\linewidth]
	\draw[black!50] (0,0) -- (50,0);
	\foreach \x/\l in {10/90,20/100,30/110,40/120}{
		\draw[black!50] (\x,0) -- +(0,-2mm) node(S)[xaxislabel]{\l\strut};
		}
	\node[axistitle,black!50,anchor=north east,yshift=-6mm,align=right] at (50,0) 
		{Zuurstofverzadiging (\%)};
	\end{tikzpicture}%
	}

\newcommand\body{%
	\begin{tikzpicture}[x=\linewidth/50]
	\useasboundingbox (80,7.8) rectangle (130,8.8);
%	\foreach \y/\l in {0/0,1/10\,000,2/20\,000,3/30\,000,4/40\,000,5/50\,000}{
%		\draw[mistgray] (0,\y) -- (11,\y);
%		}
	\foreach \x/\y in {91.1750/8.0188,98.4500/8.1827,110.7667/8.4083,112.3600/8.5360,127.7846/8.6215,123.0312/8.7359,129.2077/8.6598,108.8500/8.4184,113.7364/8.4025,93.4786/7.9183,86.4800/7.8597,83.9857/7.9695}{
		\fill (\x,\y) circle(2pt);
		}
	\end{tikzpicture}%
	}


\begin{figure}[h]
\begin{previewoutput}
%\graphtitle{Testresultaten met relatief lage variatie tussen labotesten en dus hoge ICC.}
\begin{tblr}{
	width=\linewidth,
	colspec={rX[c]},
	column{1}={leftsep=0pt},
	column{Z}={rightsep=0pt},
%	rows={rowsep=0pt},
%	columns={colsep=0pt},
%	stretch=0,
	}
\yaxis & \body \\ 
& \xaxis \\ 
\end{tblr}
\end{previewoutput}
\end{figure}











\end{document}

		

