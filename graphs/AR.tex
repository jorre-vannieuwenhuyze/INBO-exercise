% !TEX program = xelatex


\documentclass{../Jorrearticle}

\usepackage{tabularray}
\usepackage{tikz}
\input{tikzset.tex}

\usepackage[active,tightpage]{preview}
\newenvironment{previewoutput}{}{}
\PreviewEnvironment{previewoutput}
\setlength{\PreviewBorder}{0pt} 


\begin{document}






\tikzset{y=.5mm}


\newcommand\yaxis{%
	\begin{tikzpicture}[baseline=0,trim right=0,remember picture]
	\draw[black!50] (0,0) -- (0,170);
	\foreach \y/\l in {0/0,50/50,100/100/150/150}{
		\draw[black!50] (0,\y) -- +(-2mm,0) node(S)[yaxislabel]{\l\strut};
		}
	\node[axistitle,black!50,anchor=south west,xshift=-10mm,align=left] at (0,170) 
		{Zuurstofverzadiging (\%)};			
	\end{tikzpicture}%
	}
		
	
\newcommand\yaxisb{%
	\begin{tikzpicture}[baseline=0]
	\node[plotlabel,black!50,anchor=west,align=left,xshift=2pt] at (0,106.6088) {gemiddelde = 107};	
	\node[plotlabel,red!80!black,anchor=west,align=left,xshift=2pt] at (0,140) {grenswaarde = 140};
	\end{tikzpicture}%
	}	
	
	
\newcommand\xaxis{%
	\begin{tikzpicture}[x=\linewidth/11,trim left=0,trim right=\linewidth]
	\foreach \x/\l in {0/j,1/f,2/m,3/a,4/m,5/j,6/j,7/a,8/s,9/o,10/n,11/d}{
		\node at (\x,0) [xaxislabel,align=center]{\l};
		}
	\end{tikzpicture}%
	}

\newcommand\body{%
	\begin{tikzpicture}[x=\linewidth/11]
	\useasboundingbox (1,0) rectangle (12,170);
%	\foreach \y/\l in {0/0,1/10\,000,2/20\,000,3/30\,000,4/40\,000,5/50\,000}{
%		\draw[mistgray] (0,\y) -- (11,\y);
%		}
	\draw[black!50] (1,106.6088) -- +(11,0);
	\draw[red!80!black] (1,140) -- +(11,0);
	\filldraw (1,91.1750) circle(1pt) -- (2,98.4500) circle(1pt) -- (3,110.7667) circle(1pt) -- (4,112.3600) circle(1pt) -- (5,127.7846) circle(1pt) -- (6,123.0312) circle(1pt) -- (7,129.2077) circle(1pt) -- (8,108.8500) circle(1pt) -- (9,113.7364) circle(1pt) -- (10,93.4786) circle(1pt) -- (11,86.4800) circle(1pt) -- (12,83.9857) circle(1pt);
	\end{tikzpicture}%
	}


\begin{figure}[h]
\begin{previewoutput}
%\graphtitle{Testresultaten met relatief lage variatie tussen labotesten en dus hoge ICC.}
\begin{tblr}{
	width=\linewidth,
	colspec={rX[c]r},
	column{1}={leftsep=0pt},
	column{Z}={rightsep=0pt,leftsep=0pt},
	rows={rowsep=0pt},
%	columns={colsep=0pt},
%	stretch=0,
	}
\yaxis & \body & \yaxisb \\ 
& \xaxis &  \\ 
\end{tblr}
\end{previewoutput}
\end{figure}









\renewcommand\body{%
	\begin{tikzpicture}[x=\linewidth/11]
		\useasboundingbox (1,0) rectangle (12,170);
		%	\foreach \y/\l in {0/0,1/10\,000,2/20\,000,3/30\,000,4/40\,000,5/50\,000}{
			%		\draw[mistgray] (0,\y) -- (11,\y);
			%		}
		\draw[black!50] (1,106.6088) -- +(11,0);
		\draw[red!80!black] (1,140) -- +(11,0);
		\filldraw (1,87.1330) circle(1pt) -- (2,50.4598) circle(1pt) -- (3,142.6651) circle(1pt) -- (4,107.1674) circle(1pt) -- (5,159.3446) circle(1pt) -- (6,155.1313) circle(1pt) -- (7,92.2489) circle(1pt) -- (8,74.1898) circle(1pt) -- (9,107.8442) circle(1pt) -- (10,116.5891) circle(1pt) -- (11,123.8961) circle(1pt) -- (12,62.6365) circle(1pt);
	\end{tikzpicture}%
}


\begin{figure}[h]
	\begin{previewoutput}
		%\graphtitle{Testresultaten met relatief lage variatie tussen labotesten en dus hoge ICC.}
		\begin{tblr}{
				width=\linewidth,
				colspec={rX[c]r},
				column{1}={leftsep=0pt},
				column{Z}={rightsep=0pt,leftsep=0pt},
				rows={rowsep=0pt},
				%	columns={colsep=0pt},
				%	stretch=0,
			}
			\yaxis & \body & \yaxisb \\ 
			& \xaxis &  \\ 
		\end{tblr}
	\end{previewoutput}
\end{figure}




\end{document}

		

