\documentclass{Jorrearticle}

\usepackage{amsmath}
\usepackage{tcolorbox}

\title{INBO-opdracht\\Statistisch Analyse Plan}
\author{Jorre Vannieuwenhuyze}
\date{\today}

\sloppy

\begin{document}
	
\maketitle
	

\begin{tcolorbox}
Case 1:

Om de kwaliteit van waterplassen te evalueren worden verschillende
waterkwaliteitsparameters opgevolgd (Tabel 1). Gedurende één jaar wordt maandelijks
een mengstaal van de waterplas genomen. In het laboratorium wordt elke parameter
vervolgens in duplo geanalyseerd.

Omdat het onderzoeksteam jaarlijks de waterkwaliteit van meerdere plassen moet
beoordelen, wordt aan het team BMK gevraagd om het onderzoeksdesign voor de
volgende studie (gericht op één plas) te optimaliseren.

Vorig jaar werden reeds verschillende plassen onderzocht. De resultaten daarvan zijn
beschikbaar in het meegeleverde CSV‑bestand case1.csv. Voor elke waterplas wordt het
maandelijkse gemiddelde van elke parameter vergeleken met de wettelijke streef- en
grenswaarden.	
\end{tcolorbox}












Het doel van deze studie is het optimaliseren van het onderzoeksdesign
voor de jaarlijkse evaluatie van de waterkwaliteit van één waterplas.
De evaluatie gebeurt op basis van vergelijking van maandelijkse metingen
van verschillende waterkwaliteitsparameters met wettelijke streef- en
grenswaarden.

De optimalisatie van het onderzoeksdesing zoekt naar een design waarin op voldoende betrouwbare manier kan worden vastgesteld of een plas de streef- of grenswaarde overschrijdt, met zo weinig mogelijk staalnames en labkosten. Hoe meer staalnames worden getrokken en hoe meer deze staalnames geanalyseerd worden, hoe betrouwbaarder de resultaten maar ook hoe hoger de kosten.

Concreet beoogt dit analyseplan:

\begin{itemize}
	\item Het kwantificeren van de verschillende variantiecomponenten
	(maand-tot-maand variatie versus laboratoriumvariatie);
	\item Het bepalen van de minimale meetfrequentie die vereist is om
	overschrijdingen met voldoende statistische zekerheid te detecteren;
	\item Het formuleren van een modelgebaseerde evaluatiemethode
	voor jaarlijkse kwaliteitsbeoordeling.
\end{itemize}
%
%\subsection{Datastructuur}
%
%Voor elke parameter geldt:
%
%\begin{itemize}
%	\item Maandelijkse bemonstering ($m = 1,\dots,12$);
%	\item Laboratoriumanalyse in duplo ($r = 1,2$);
%	\item Waarneming $Y_{m,r}$.
%\end{itemize}
%
%Voor elke parameter wordt een wettelijke grenswaarde $T$ gedefinieerd.





















\section{Stap 1: Waar komt de variatie vandaan?}

%Resultaat van een test op staalname = De echte waarde in maand $m$ + een willekeurige fout door steekproeftrekking van de staalname + willekeurige fout door test in lab
%
%
%We berekenen voor elke parameter de Intraclass Correlation Coefficient (ICC)


Elke maand wordt een staal genomen uit de waterplas.
Dat staal wordt in het labo twee keer geanalyseerd (duplo-analyse).

Wanneer we verschillen zien tussen metingen,
kunnen die uit verschillende bronnen komen:

\begin{itemize}
	\item Echte verschillen tussen maanden (bijvoorbeeld door weersomstandigheden of biologische processen);
	\item Kleine verschillen die ontstaan bij het nemen van het staal (bemonsteringsvariatie);
	\item Meetonnauwkeurigheid in het labo (laboratoriumvariatie).
\end{itemize}

Met andere woorden: niet elk verschil in meetresultaat betekent
dat de waterkwaliteit echt veranderd is.
Een deel van de variatie is simpelweg toe te schrijven aan toeval
bij staalname of analyse.

In deze stap onderzoeken we hoeveel van de totale variatie
wordt veroorzaakt door:

\begin{itemize}
	\item Echte verschillen tussen maanden;
	\item Willekeurige meetfouten (staalname + labo).
\end{itemize}

Daarvoor berekenen we de zogenaamde
\textbf{Intraclass Correlation Coefficient (ICC)}.

De ICC geeft aan welk deel van de totale variatie
toe te schrijven is aan echte verschillen tussen maanden.

\begin{itemize}
	\item Als de ICC hoog is (bijvoorbeeld hoger dan 0.80),
	betekent dit dat bijna alle variatie echte maandverschillen weerspiegelt.
	De bijdrage van meetfouten is dan beperkt,
	en duplo-analyses voegen weinig extra informatie toe.
	
	\item Als de ICC laag is (bijvoorbeeld lager dan 0.50),
	betekent dit dat een groot deel van de variatie
	afkomstig is van staalname- of labo-onzekerheid.
	In dat geval zijn duplo-analyses belangrijk om betrouwbare resultaten te verkrijgen.
\end{itemize}

Deze analyse laat toe om objectief te beoordelen
of dubbele labo-analyses noodzakelijk zijn,
of dat dezelfde betrouwbaarheid kan worden bereikt
met minder metingen en dus lagere kosten.










\section{Stap 2: Zijn er seizoenseffecten?}

Waterkwaliteit kan verschillen tussen winter en zomer.
Daarom onderzoeken we of bepaalde maanden systematisch hogere of lagere waarden vertonen.

Indien blijkt dat:

\begin{itemize}
	\item Sommige maanden duidelijk risicovoller zijn dan andere,
\end{itemize}

dan kan het meetplan worden aangepast, bijvoorbeeld:

\begin{itemize}
	\item Vaker meten in de zomer;
	\item Minder frequent meten in stabiele wintermaanden.
\end{itemize}

Zo richten we de inspanningen waar het risico het grootst is.

\subsection{Stap 3: Jaarlijkse beoordeling op basis van het geheel}

Momenteel wordt vaak per maand gekeken of een grenswaarde wordt overschreden.
Dat kan leiden tot sterke schommelingen in beoordeling.

In plaats daarvan bekijken we het volledige jaar in samenhang.
We schatten het gemiddelde niveau van de parameter over het hele jaar,
rekening houdend met seizoenseffecten.

Daarna beoordelen we:

\begin{itemize}
	\item Ligt het geschatte jaargemiddelde duidelijk boven de wettelijke grens?
\end{itemize}

Pas wanneer we met voldoende zekerheid kunnen zeggen
dat het gemiddelde boven de grens ligt,
spreken we van een overschrijding.

Dit voorkomt dat één toevallige piek meteen tot een negatieve beoordeling leidt.

\subsection{Stap 4: Hoe vaak moeten we meten?}

Meer metingen geven meer zekerheid,
maar kosten ook meer geld.

We berekenen daarom hoeveel metingen nodig zijn
om een relevante overschrijding met voldoende zekerheid te kunnen vaststellen.

We vergelijken verschillende scenario’s:

\begin{itemize}
	\item 12 maanden met telkens 2 labo-analyses (huidige situatie);
	\item 12 maanden met 1 analyse;
	\item 8 maanden met 1 analyse;
	\item Seizoensgericht meten.
\end{itemize}

Voor elk scenario bekijken we:

\begin{itemize}
	\item Hoe zeker is de beoordeling?
	\item Hoe groot is de kans dat we een echte overschrijding missen?
	\item Wat is de totale kost?
\end{itemize}

Zo zoeken we het beste evenwicht tussen betrouwbaarheid en kostenefficiëntie.

\subsection{Stap 5: Flexibel opvolgen tijdens het jaar}

Het meetplan hoeft niet volledig vast te liggen.

Indien tijdens het jaar blijkt dat:

\begin{itemize}
	\item De waarden sterk stijgen;
	\item Of we dicht bij een grenswaarde komen;
\end{itemize}

dan kan tijdelijk vaker worden gemeten.

Wanneer de situatie opnieuw stabiel is,
kan men terugkeren naar het standaardmeetplan.

Dit noemen we adaptieve monitoring:
we meten intensiever wanneer dat nodig is.

\subsection{Beslissingscriteria}

Een aangepast meetplan wordt enkel ingevoerd indien:

\begin{itemize}
	\item De betrouwbaarheid van de beoordeling nauwelijks daalt;
	\item De kans om een echte overschrijding te missen beperkt blijft;
	\item De kosten duidelijk verminderen.
\end{itemize}

\subsection{Rapportering}

Voor elke parameter wordt gerapporteerd:

\begin{itemize}
	\item Hoe groot de natuurlijke maandelijkse schommelingen zijn;
	\item Hoe groot de meetonnauwkeurigheid van het labo is;
	\item Of er duidelijke seizoenseffecten zijn;
	\item Het geschatte jaargemiddelde met aanduiding van onzekerheid;
	\item De vergelijking tussen de verschillende meetscenario’s;
	\item Het aanbevolen meetplan.
\end{itemize}
















\section{Stap 3: Rekening houden met samenhang tussen waterkwaliteitsparameters}

\subsection{Waarom is dit belangrijk?}

In de huidige aanpak wordt elke waterkwaliteitsparameter afzonderlijk
gemeten en beoordeeld. In werkelijkheid hangen verschillende
waterkwaliteitsparameters echter sterk met elkaar samen.

Bijvoorbeeld:

\begin{itemize}
	\item Nutriëntenconcentraties hangen vaak samen met algenbloei;
	\item Zuurstofgehalte hangt samen met biologische activiteit;
	\item Troebelheid kan samenhangen met fosfaat- of stikstofgehalte.
\end{itemize}

Wanneer twee parameters sterk samen bewegen in de tijd,
bevat de ene parameter gedeeltelijk informatie over de andere.
Dit biedt mogelijkheden om het onderzoeksdesign efficiënter te maken.

\subsection{Het kernidee}

Stel:

\begin{itemize}
	\item Parameter A is goedkoop om te meten;
	\item Parameter B is duur om te analyseren;
	\item A en B vertonen een sterke en stabiele samenhang.
\end{itemize}

In dat geval kan men:

\begin{itemize}
	\item Parameter A maandelijks meten;
	\item Parameter B minder frequent meten (bv. per kwartaal);
	\item De tussentijdse waarden van B inschatten op basis van A.
\end{itemize}

Hierdoor dalen de analysekosten,
terwijl de informatie over de waterkwaliteit grotendeels behouden blijft.

\subsection{Hoe bepalen we of dit mogelijk is?}

Op basis van historische gegevens wordt nagegaan:

\begin{enumerate}
	\item Bewegen bepaalde parameters systematisch samen?
	\item Is deze samenhang stabiel doorheen het jaar?
	\item Hoe goed kan de ene parameter voorspeld worden op basis van de andere?
\end{enumerate}

Wanneer hoge waarden van de ene parameter bijna altijd samengaan
met hoge waarden van een andere parameter (en lage waarden eveneens),
spreken we van een sterke samenhang.

Indien deze relatie voldoende stabiel is,
kan een deel van de informatie van de dure parameter
worden afgeleid uit de goedkopere parameter.

\subsection{Wat betekent dit concreet voor het meetdesign?}

In plaats van alle parameters maandelijks te meten,
kan het meetplan worden aangepast, bijvoorbeeld:

\begin{itemize}
	\item Goedkope parameters: maandelijkse meting;
	\item Dure parameters: meting per kwartaal;
	\item Jaarlijkse controle of de onderlinge samenhang stabiel blijft.
\end{itemize}

\subsection{Voorwaarden en waarborgen}

Een reductie van meetfrequentie is enkel verantwoord indien:

\begin{itemize}
	\item De samenhang tussen parameters sterk en stabiel is;
	\item De kans op een foutieve beoordeling van de waterkwaliteit
	beperkt blijft;
	\item Het model jaarlijks wordt geëvalueerd en indien nodig bijgestuurd.
\end{itemize}

Indien tijdens het jaar blijkt dat de parameters minder sterk samenhangen
dan verwacht, kan tijdelijk opnieuw een hogere meetfrequentie worden ingevoerd.

\subsection{Voordeel van deze aanpak}

Door gebruik te maken van de samenhang tussen parameters:

\begin{itemize}
	\item Worden middelen efficiënter ingezet;
	\item Kan het aantal dure labo-analyses worden beperkt;
	\item Blijft de betrouwbaarheid van de kwaliteitsbeoordeling gewaarborgd.
\end{itemize}

Deze aanpak combineert kostenefficiëntie met wetenschappelijke onderbouwing.












%
%
%
%
%\subsection{Stap 1: Variantiecomponentenanalyse}
%
%Om de efficiëntie van duplo-analyses te evalueren,
%wordt per parameter een lineair mixed model geschat:
%
%\begin{equation}
%Y_{m,r} = \mu + \alpha_m + \varepsilon_{m,r}
%\end{equation}
%
%
%met:
%
%\[
%\alpha_m \sim \mathcal{N}(0,\sigma^2_{\text{maand}})
%\]
%\[
%\varepsilon_{m,r} \sim \mathcal{N}(0,\sigma^2_{\text{lab}})
%\]
%
%waar:
%
%\begin{itemize}
%	\item $\sigma^2_{\text{maand}}$ = maand-tot-maand variatie
%	\item $\sigma^2_{\text{lab}}$ = laboratoriumvariatie
%\end{itemize}
%
%De intraclass correlatiecoëfficiënt (ICC) wordt berekend als:
%
%\[
%ICC = \frac{\sigma^2_{\text{maand}}}
%{\sigma^2_{\text{maand}} + \sigma^2_{\text{lab}}}
%\]
%
%Interpretatie:
%
%\begin{itemize}
%	\item $ICC > 0.80$: beperkte meerwaarde van duplo-analyses
%	\item $ICC < 0.50$: significante labvariatie, behoud duplo’s aanbevolen
%\end{itemize}
%
%
%
%\subsection{Stap 2: Seizoensstructuur}
%
%Seizoenseffecten worden onderzocht via:
%
%\[
%Y_m = \mu + \gamma_{s(m)} + \varepsilon_m
%\]
%
%waar $s(m)$ de maand of seizoen aanduidt.
%
%Hypothese:
%
%\[
%H_0: \gamma_1 = \gamma_2 = \dots = \gamma_{12}
%\]
%
%Indien significante seizoensstructuur aanwezig is,
%kan een gedifferentieerd bemonsteringsschema overwogen worden
%(bv. hogere frequentie in risicomaanden).
%
%\subsection{Stap 3: Modelgebaseerde evaluatie van grenswaarde}
%
%In plaats van maandelijkse puntvergelijkingen,
%wordt een modelgebaseerde schatting van het jaargemiddelde $\mu$ gebruikt.
%
%Het model:
%
%\[
%Y_t = \mu + s(t) + \varepsilon_t
%\]
%
%met $s(t)$ een seizoenscomponent.
%
%De evaluatie gebeurt via:
%
%\[
%H_0: \mu \leq T
%\quad \text{vs} \quad
%H_1: \mu > T
%\]
%
%Beslissingsregel:
%
%Overschrijding wordt vastgesteld indien de ondergrens
%van het $(1-\alpha)$ betrouwbaarheidsinterval voor $\mu$
%boven $T$ ligt.
%
%\subsection{Stap 4: Power-analyse en optimalisatie meetfrequentie}
%
%De vereiste steekproefomvang (aantal maanden $n$)
%om een relevante overschrijding $\delta$ te detecteren
%met significantieniveau $\alpha$ en power $1-\beta$ wordt benaderd via:
%
%\[
%n =
%\frac{(z_{1-\alpha} + z_{1-\beta})^2 \sigma^2}
%{\delta^2}
%\]
%
%waar:
%
%\begin{itemize}
%	\item $\sigma^2$ = geschatte maandvariatie
%	\item $\delta$ = minimaal relevante afwijking t.o.v. $T$
%\end{itemize}
%
%Scenario-analyse wordt uitgevoerd voor:
%
%\begin{itemize}
%	\item 12 maanden × 2 analyses (huidig design)
%	\item 12 maanden × 1 analyse
%	\item 8 maanden × 1 analyse
%	\item Seizoensgericht design
%\end{itemize}
%
%Voor elk scenario worden berekend:
%
%\begin{itemize}
%	\item Standaardfout van $\hat{\mu}$
%	\item Power om $\delta$ te detecteren
%	\item Totale analysekost
%\end{itemize}
%
%\subsection{Stap 5: Adaptieve monitoring}
%
%Indien tijdens het lopende jaar:
%
%\begin{itemize}
%	\item Een significante positieve trend wordt vastgesteld;
%	\item Of de geschatte kans op overschrijding
%	$P(\mu > T)$ groter is dan een vooraf bepaalde drempel (bv. 0.30);
%\end{itemize}
%
%kan tijdelijk een verhoogde meetfrequentie worden ingevoerd.
%
%\subsection{Beslissingscriteria voor optimalisatie}
%
%Een alternatief design wordt weerhouden indien:
%
%\begin{enumerate}
%	\item De power om relevante overschrijdingen te detecteren
%	niet meer dan 5\% daalt t.o.v. het huidige design;
%	\item De standaardfout van het jaargemiddelde
%	minder dan 10\% toeneemt;
%	\item De totale analysekost significant gereduceerd wordt.
%\end{enumerate}
%
%\subsection{Rapportering}
%
%Voor elke parameter worden gerapporteerd:
%
%\begin{itemize}
%	\item Geschatte variantiecomponenten;
%	\item ICC;
%	\item Seizoenseffecten;
%	\item Geschat jaargemiddelde met betrouwbaarheidsinterval;
%	\item Power per scenario;
%	\item Aanbevolen onderzoeksdesign.
%\end{itemize}



	
\end{document}